\section{Primer Dia: Sisifo Observa la Montaña}
\emph{08 de Mayo de 2025}

Llevo programando ya casi 5 años y de ellos los proyectos mas largos que he hecho
son como maximo de 2000 lineas de codigo. Creo que se me da bien pero la verdad es que
ya estoy algo cansado de los proyectos pequeños. Por lo tanto hoy asumi un nuevo reto.

Voy a hacer un lenguaje de programación con el que hacer metodos formales. Debo ser
honesto, el titulo de esta entrada es, cuando mucho, un deseo. No soy Sisifo viendo
la montaña que se me viene. Pero intentar acotar un proyecto asi de grande antes de
siquiera iniciar suele ser una gran
excusa para procrastinar (y creanme que en eso si soy experto). Asi que no hay mas que mirar
que simplemente ponerse los guantes y salir a ver que se encuentra.

Con esta mentalidad decidi que lo primero que queria hacer era aprender a programar un
lenguaje perse. En este momento debo sincerarme. Nunca he hecho un compilador. He hecho
transpiladores (y varios de ellos) ademas de un par de interpretes (entre los que estan
mas interpretes de BrainF*ck de lo que suele ser una buena idea). Por lo tanto me siento
relativamente tranquilo con el front-end de un compilador. Ahora bien, el back-end es para
mi un misterio aun. Por lo tanto doy por iniciado con este articulo la etapa de exploración
de este proyecto. Esta etapa consistira en programar la mayor cantidad de compiladores que
pueda para considerar diferentes posibilidades. Iniciare por un tutorial para crear scheme
\footnote{\url{https://en.wikibooks.org/wiki/Write_Yourself_a_Scheme_in_48_Hours}} en
Haskell solamente para probar y seguire asi durante un mes. El objetivo es en este mes
haber hecho por lo menos 3 proyectos distintos en donde haya construido 3 lenguajes de
programación diferentes. Como se imaginaran esto tomara mucho tiempo por cada uno asi que
tengo que comprometerme. Ya el dia de hoy inicie un
poco\footnote{Se que suena sospechoso pero aun no pienso subirlo a github hasta que no avance otro poco aun cuando ya estoy usando git para hacer su seguimiento. Cuando lo sienta en mejores condiciones lo subire}
con un parser para Scheme pero aun estoy lejos de conseguir algo interesante.

Ahora bien, quiero confesar que este no va a ser el unico medio por el que lo pondre. Mi objetivo es
cada semana (el fin de semana) grabar un video en el cual cuento que hice esa semana y por que me parece
muy interesante. No creo que sea un video particularmente producido aunque quien sabe si me de la
chiripiorca despues. A fin de cuentas este es un proyecto solamente por aprender asi que no me niego
a nada de lo que el viento me traiga. Hoy es miercoles asi que aun tengo un poco de tiempo para
el primer video pero la verdad es que las ganas de que esto salga bien me mueven todo el cuerpo.

Espero seguir contandoles mas a medida que el tiempo pasa. Nos vemos despues.
